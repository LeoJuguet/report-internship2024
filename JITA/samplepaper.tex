% This is samplepaper.tex, a sample chapter demonstrating the
% LLNCS macro package for Springer Computer Science proceedings;
% Version 2.21 of 2022/01/12
%
\documentclass[runningheads]{llncs}
%
\usepackage[T1]{fontenc}
% T1 fonts will be used to generate the final print and online PDFs,
% so please use T1 fonts in your manuscript whenever possible.
% Other font encondings may result in incorrect characters.
%
\usepackage{graphicx}
% Used for displaying a sample figure. If possible, figure files should
% be included in EPS format.
%
% If you use the hyperref package, please uncomment the following two lines
% to display URLs in blue roman font according to Springer's eBook style:
% \usepackage{color}
% \renewcommand\UrlFont{\color{blue}\rmfamily}
% \urlstyle{rm}
%
\begin{document}
%
\title{
  Jasa : A new Jasmin Safety Analyser based on MOPSA
}
%
%\titlerunning{Abbreviated paper title}
% If the paper title is too long for the running head, you can set
% an abbreviated paper title here
%
\author{Léo Juguet\inst{1,2} \and
		Manuel Barbosa\inst{3,4} \and
		Benjamin Grégoire\inst{5} \and
		Vincent Laporte\inst{6} \and
		Raphaël Monat\inst{7}
}
%
\authorrunning{L. Juguet et al.}
% First names are abbreviated in the running head.
% If there are more than two authors, 'et al.' is used.
%
\institute{ENS Paris-Saclay \email{leo.juguet@ens-paris-saclay.fr} \and
           MPI-SP, Bochum, Germany \and
		   University of Porto (FCUP), Porto, Portugal \email{mbb@fc.up.pt}\and
		   INESC TEC, Porto, Portugal \and
		   Université Côte d'Azur, Inria, France \email{benjamin.gregoire@inria.fr} \and
		   Université de Lorraine, CNRS, Inria, LORIA, F-54000 Nancy, France \email{vincent.laporte@inria.fr} \and
		   Univ. Lille, Inria, CNRS, Centrale Lille, UMR 9189 CRIStAL, F-59000 Lille, France \email{raphael.monat@inria.fr}
		}
%
\maketitle              % typeset the header of the contribution
%
\begin{abstract}
We present a new security checker for Jasmin, a programming language for 
implementing high-assurance, high-speed cryptographic codes. The safety
checker detects scalar initialization, array initialization to some extent, 
and out-of-bounds access without unrolling loops using widening. This new safety 
checker is mainly based on the MOPSA library. 
It is undeniably faster than its predecessor, 
thanks to support for function contracts and modular analysis. 
This new safety controller, created with MOPSA, also allows more properties 
to be checked than the previous safety controller, with the ability to analyse 
values.

\keywords{Abstract Interpretation \and
		  Array abstraction \and 
		  Array content analysis \and 
		  Program verification \and 
		  Static analysis}
\end{abstract}
%
%
%
\section{Resume}
\subsection{Introduction}

\subsubsection{Context} 
Writing a secure program is hard, as there are many considerations to take into 
account, and often small mistakes are 
involuntarily made by programmers, such as badly definining variables or 
accessing out-of-bound memory. These mistakes 
can lead to writing at unsafe locations, leading to an information leak. 
Hence, a tool is needed to detect mistakes.

In particular for Jasmin \cite{jasmin} a programming language that aims 
to be secure for cryptographic code.
Its compiler is mostly written in Coq and provides a proof that the code will be 
correctly translated to assembly, under certain assumptions.
These assumptions are checked by a safety checker,
that use abstract interpretation \cite{cousot-lattice}
 but the current tool is too slow, and not as precise as we wanted. Moreover
the safety checker is hard to maintain today, and doesn't allow modular analysis.

The goal of this work was to create a new safety checker
more efficient, more maintanable, and 
more precise safety analyzer that would be able to check that safety conditions of 
any Jasmin code are verified.


\subsubsection{Contribution}
We wrote a new safety checker for Jasmin 
\footnote{The source code can be found here : \url{https://github.com/LeoJuguet/jasmin/tree/cryptoline-mopsa/compiler/jasa}}\cite{jasmin} 
using the MOPSA \cite{mopsa-phd} library . 
MOPSA\cite{mopsa-phd} 
is a modular open platform for static analysis that uses abstract interpretation 
and is designed to be modular, 
so that a frontend for different languages can easily be added. By relying on 
the third-party MOPSA library, which maintains the abstract interpretation 
backend, the resulting code base is easier to maintain from Jasmin's point 
of view.




This new safety checker can check whether scalars and arrays are 
correctly initialized. The analysis of array initialisation is based on 
an algorithm mainly inspired by a paper by Cousot \cite{cousot-article}, 
which cannot 
currently be implemented in MOPSA. This algorithm takes into account 
two main criteria: a limited number of variables needed to represent 
an array and it is not necessary to unfold loops when analysing arrays. In the 
end, all arrays are represented by three segments and the abstraction 
of cell values is not limited to an initialisation domain.

In addition, this safety checker can detect out-of-bounds errors 
in arrays and is modular, allowing function-independent analysis. 
Thanks to contract support for checking properties, although at present 
only contracts for array initialisation have been implemented. It also works 
faster than the current checker.


\paragraph{Limits}
However, for the moment, the tool cannot be used in a complete Jasmin code 
base due to the current lack of support for special instructions and memory.
But support for special instructions can easily be added with the work already 
done on function calls. Memory analysis is more difficult because memory aliasing 
has to be taken into account.


\begin{credits}
\subsubsection{\ackname}
This work was produced during 5-month internship at MPI-SP,
in the Jasmin compiler team.


\subsubsection{\discintname}
The authors have no competing interests to declare that are
relevant to the content of this article.

\end{credits}
%
% ---- Bibliography ----
%
% BibTeX users should specify bibliography style 'splncs04'.
% References will then be sorted and formatted in the correct style.
%
\bibliographystyle{splncs04}
\bibliography{bibliography.bib}
%
% \begin{thebibliography}{8}
% \bibitem{ref_article1}
% Author, F.: Article title. Journal \textbf{2}(5), 99--110 (2016)

% \bibitem{ref_lncs1}
% Author, F., Author, S.: Title of a proceedings paper. In: Editor,
% F., Editor, S. (eds.) CONFERENCE 2016, LNCS, vol. 9999, pp. 1--13.
% Springer, Heidelberg (2016). \doi{10.10007/1234567890}

% \bibitem{ref_book1}
% Author, F., Author, S., Author, T.: Book title. 2nd edn. Publisher,
% Location (1999)

% \bibitem{ref_proc1}
% Author, A.-B.: Contribution title. In: 9th International Proceedings
% on Proceedings, pp. 1--2. Publisher, Location (2010)

% \bibitem{ref_url1}
% LNCS Homepage, \url{http://www.springer.com/lncs}, last accessed 2023/10/25
% \end{thebibliography}
\end{document}
